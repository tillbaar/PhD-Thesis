% ┌────────────────────────────────────────────────────────────────────────────┐
% │ ALU PAPER                                                                  │
% │   Evolution shapes the RNA metabolism of Alu retrotransposons              │
% └────────────────────────────────────────────────────────────────────────────┘

\chapter{Evolution Shapes the Alu RNA Metabolism}

\section{Project Introduction}

\subsection{RNAseq Data}

\section{Failed Explorations}

\section{Differential Expression Analysis}

\section{Half-Life Estimation}

\vfill
\noindent My contribution to this publication was the complete bioinformatic
and statistical analysis.\nopagebreak
\medskip
\begin{tcolorbox}[
  boxrule=0pt, leftrule=1pt, colframe=s-blue, colback=white, sharp corners=all]%
  \raggedright
  Baar, T., Dümcke, S., Gressel, S., Schwalb, B.,
  Dilthey, A., Cramer, P., Tresch, A. (2022).
  
  \smallskip
  \href{https://doi.org/10.1093/g3journal/jkac054}
    {RNA transcription and degradation of Alu retrotransposons depends on
    sequence features and evolutionary history}

  \smallskip
  \textit{G3: Genes}\thinspace{}|\thinspace{}\textit{Genomes}%
    \thinspace{}|\thinspace{}\textit{Genetics, ???}
\end{tcolorbox}