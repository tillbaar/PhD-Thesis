% ┌────────────────────────────────────────────────────────────────────────────┐
% │ INTRODUCTION                                                               │
% └────────────────────────────────────────────────────────────────────────────┘

\chapter{Introduction}
\sectionmark{Introduction} % This sets the outer header mark to "Introduction"
%                          % until the first section is begun.

\enquote{The temptation to form premature theories upon insufficient data is 
the bane of our profession.} [Sherlock Holmes, \citealt{Holmes}]. This 
statement applies not only to fictional consulting detectives but likewise to 
the field of science. One might even say that it is an inherent tendency of 
the human mind to jump to conclusions based on incomplete knowledge. Thus, it
is the statistician's task to rigidify all conjectures made in the context of
a scientific investigation and base them firmly on observations. This is, of
course, easier said than done.

In general, the art of the statistician, in contrast to purely technical 
aspects, is to specify the model according to the data at hand. Upon contact 
with data, every mathematical model suffers from the bias-variance tradeoff
\citep{VonLuxburg2009}, i.e., a model needs to find the right balance between 
overfitting (high variance) and underfitting (high bias). Bias, in this case, 
describes the difference between a model's average predictions and the true 
values. A model with high bias is oversimplified. Variance refers to the 
variability of a model's predictions given the true values. A model with high 
variance fails to generalise and cannot make accurate predictions.
\begin{align*}
  \intertext{Following an example by \citealt{James2009}, assume we wish to
  predict an outcome $Y$ given observations $X=\left\{x_1\dots x_n\right\}$
  with a relationship of}
  Y =&\, f\!\left(X\right)+\varepsilon
  \intertext{where $\varepsilon$ describes a normally distributed error term
  with mean 0. Further assuming a model $\widehat{f}\!\left(X\right)$ of
  $f\!\left(X\right)$, the expected squared error for a point $x$ becomes}
  \operatorname{E}\left(x\right) =&\,
    \operatorname{E}\left[\left(Y-\widehat{f}\!\left(x\right)\right)^2\right]
  \intertext{with $\operatorname{E}$ denoting the expected value. This further
  decomposes to}
  \operatorname{E}\left(x\right) =& 
     \underbracket{\vphantom{\left[\left(\widehat{f}\right)^2\right]}
     \left(\operatorname{E}\left[\widehat{f}\!\left(x\right)\right]-
      f\!\left(x\right)\right)^2}_{\text{Bias}^2}+
      \underbracket{\operatorname{E}\left[\left(\widehat{f}\!\left(x\right)-
      \operatorname{E}\left[\widehat{f}\!\left(x\right)\right]\right)^2\right]}
      _{\text{Variance}}+
      \,\sigma^2_\varepsilon
  \shortintertext{where $\sigma^2_\varepsilon$ is the irreducible error,
  i.e., the amount of inherent noise in the data that cannot be removed.}
\end{align*}
% To reduce the vertical spacing between the align environment and the
% following text, use \shortintertext instead of \intertext for the last
% paragraph and follow the align environment with the command:
% \vspace{-\the\belowdisplayshortskip} to remove unnecessary space.
% ──────────────────────────────────────────────────────────────────────
% Ugly, I know, but it works and doesn't require too much effort.

\vspace{-\the\belowdisplayshortskip}
\noindent In other words, a model that does not capture the pattern from which 
the observations emerge is underfitted, usually exhibiting high bias and low 
variance. Vice versa, a model that captures the noisiness of the observations
alongside the pattern is overfitted, usually exhibiting low bias and high
variance.

The bias-variance tradeoff is connected to the complexity of a model. A model 
that is too simple and thus undercomplex for the data will underfit. This is 
because it has too few parameters to model the data adequately. Conversely, a 
model that is overcomplex will overfit, as it has too many parameters. Of
course, one would hope to optimally balance each model's complexity, bias, and
variance so to never over- or underfit, or at least come feasible close to
this goal.

To approach the optimal model, commonly used methods are regularisation,
boosting, and bagging.
\bigbreak

\noindent Regularisation \marginnote{Regularisation} aims to mitigate the 
problem of overfitting common to overcomplex models \citep{Deisenroth2020}. 
Assume again a model that predicts $f\!\left(X\right)$ and possesses many 
parameters $\theta_1\dots\theta_n$. Its associated loss function $V$ governs 
the training of the model \citep{Rosasco2003}. Regularisation adds a 
regularisation term or regulariser $R$ to the loss function that penalises 
complexity of the model. Thus, the expression to be minimised becomes
\begin{align*}
  \min_{\widehat{f}}\sum_{i=1}^n
  V\!\left(\widehat{f}\!\left(x_i\right),f\!\left(x_i\right)\right)+
  \lambda\:R\!\left(\widehat{f}\:\right)
\end{align*}
with the parameter $\lambda$ controlling the amount of regularisation that is
applied.

Two common applications of regularisation are the linear regression techniques 
Ridge regression \citep{Hoerl1970} and Lasso regression \citep{Tibshirani1996}.
While these can be extended to other statistical models, assume simple linear
regression for the sake of demonstration. Both techniques add a regularisation
term to the loss function that depends directly on the values of the model's
parameters $\theta_i$
\begin{align*}
  \underset{\text{Ridge regression}}{
    R\left(\theta_{i}\right)=\lambda\sum_{i=1}^{n}\theta_{\,i}^{\,2}
  }
  \quad\text{and}\quad
  \underset{\text{Lasso regression}}{
    R\left(\theta_{i}\right)=\lambda\sum_{i=1}^{n}\left|\,\theta_{\,i}\right|
  }
\end{align*}
thus shrinking all but the most influential parameters of the model and 
thereby reducing model complexity and multicollinearity \citep{Herawati2018}. 
The difference between Ridge and Lasso regression lies in the calculation of 
the applied penalty. While Ridge regression penalises the sum of the squared 
coefficients (L2 penalty), Lasso regression penalises the sum of their 
absolute values (L1 penalty). The ultimate consequence is that while Lasso can 
shrink non-influential parameters to zero, Ridge cannot. On the other hand, 
this can cause Lasso to eliminate important parameters under 
multicollinearity, if predictor variables are correlated, as it tends to 
select one parameter from the correlated group and ignore the rest.

To overcome these limitations, a combination of Ridge and Lasso regression can 
be applied, elastic net \citep{Zou2005}. The used regularisation technique 
combines an L1 and an L2 penalty by using separate $\lambda$ parameters for 
each, $\lambda_{1}$ and $\lambda_{2}$. If $\lambda_{1}=0$, the penalty equals 
Ridge regularisation; if $\lambda_{2}=0$, the penalty equals Lasso 
regularisation; and if $\lambda_{1}>0$ and $\lambda_{2}>0$, a combination of 
both is applied.
\bigbreak

\noindent While \marginnote{Boosting} regularisation is a helpful method to 
deal with overcomplex models, boosting addresses the problem of poor models in 
more general terms \citep{Freund1999}; a \textquotesingle{}poor 
model\textquotesingle{}, in this case, refers to a weak learner. 
\citet{Valiant} formalised the concept of learnability in the context of 
computational complexity theory and introduced the probably approximately 
correct (PAC) model. A problem is PAC-learnable if there exists a model that, 
with a chance higher than a threshold $\delta$, will arrive at a solution with 
a generalisation error smaller than a threshold $\epsilon$. Generalisation 
error or out-of-sample error refers to a model's predictive performance on 
previously unseen data \citep{Bousquet2004}. A model satisfying these 
conditions for any given problem is called a strong learner, while a model 
that does not is a weak learner.

A problem can benefit from boosting if applying a strong learner is either 
impossible, as no strong learner exists, or disadvantageous, for example, 
because the strong learner is prohibitively complex and thus underperformant, 
or because the available training data is insufficient to apply it. While 
current machine learning research, especially in the field of deep learning 
\citep{LeCun2015}, mainly approaches the challenge of more complex problems by 
fielding stronger algorithms, boosting seeks to improve the results of weak 
learners.

While \citet{Kearns} defined weak learners as models that perform just 
slightly better than random guessing, \citet{Schapire1990} demonstrated their 
power if applied correctly, proving that any problem solvable by a strong 
learner is equally solvable by a collection of weak learners: the hypothesis 
boosting mechanism. The term \textquotesingle{}hypothesis\textquotesingle{} 
here describes the solution a model arrives at after training, the model's 
final parameters. \citet{Freund1995} improved this further, combining many 
weak learners and using their combined results to arrive at a strong 
prediction, one weak learner effectively compensating for the shortcomings of 
another. The next step was AdaBoost, adaptive boosting, for which Freund and 
Schapire were awarded the Gödel Prize in 2003 \citep{Freund1997}. This 
boosting variant scales each weak learner's influence on the final prediction 
depending on their own error. The current state-of-the-art boosting technique 
is gradient boosting with its predominant implementation XGBoost 
\citep{Chen2016}. In contrast to AdaBoost, which always minimises the 
exponential loss function, gradient boosting can use any differentiable loss 
function, which makes it adaptable to many classification and regression tasks.

Following an example by \citet{Li}, assume again a model
$\widehat{f}\!\left( X \right)$. The boosting model iterates over
$M$\vphantom{$\widehat{f}$} stages, and each stage $m$ has an associated
imperfect model $\widehat{f}\!\left( X \right)_{m}$, so that at each stage, a
new \textquotesingle{}hypothesis\textquotesingle{} is added, a new estimator
$\widehat{g}\left(X\right)_{m}$\vphantom{$\widehat{f}$}.
\begin{align*}
  \widehat{f}\!\left(X\right)_{m+1}&=\widehat{f}\!\left(X\right)_{m}+
    \widehat{g}\left(X\right)_{m}
  \intertext{As the model iterates over sets of training data
  $Y_{i}=\left\{y_1\dots y_n\right\}$, the new estimator is fit to the
  residual, the difference between the values of the training data $Y_{i}$ and
  the estimation of the previous model.}
  \widehat{g}\left(X\right)_{m}&=Y_{i}-\widehat{f}\!\left(X\right)_{m}
\end{align*}
In this fashion, each new stage $m+1$ attempts to correct for the errors made
 by the previous stage $m$.
\bigbreak

\noindent While \marginnote{Bagging} regularisation



\vspace{5cm}
\subsubsection{Regression}

\section{Method Overview}







































% ┌────────────────────────────────────────────────────────────────────────────┐
% │ ALU PAPER                                                                  │
% │   Evolution shapes the RNA metabolism of Alu retrotransposons              │
% └────────────────────────────────────────────────────────────────────────────┘

\chapter{Evolution Shapes the Alu RNA Metabolism}

\section{Project Introduction}

\subsection{RNAseq Data}

\section{Failed Explorations}

\section{Differential Expression Analysis}

\section{Half-Life Estimation}

\begin{tcolorbox}[
  boxrule=0pt, leftrule=1pt, colframe=s-blue, colback=white, sharp corners=all]%
  \raggedright
  Baar, T., Dümcke, S., Gressel, S., Schwalb, B.,
  Dilthey, A., Cramer, P., Tresch, A. (2022).
  
  \smallskip
  \href{http://www.overleaf.com}
    {Evolution shapes the RNA metabolism of Alu retrotransposons}

  \smallskip
  \textit{Seminars in cell \& developmental biology, 22(1):39–47.}
\end{tcolorbox}

% ┌────────────────────────────────────────────────────────────────────────────┐
% │ ANGIO PAPER                                                                │
% │   Endoscopic hemostasis makes the difference: Angiographic treatment in    │
% │   patients with lower gastrointestinal bleeding                            │
% └────────────────────────────────────────────────────────────────────────────┘

\chapter{Angiography for Gastrointestinal Bleeding}

\section{Project Introduction}

\section{Failed Explorations}

\section{Decision Trees for Diagnostic Support}

\begin{tcolorbox}[
  boxrule=0pt, leftrule=1pt, colframe=s-blue, colback=white, sharp corners=all]%
  \raggedright
  Werner, DJ., Baar, T., Kiesslich, R., Wenzel, N., Abusalim, N., Tresch, A.,
  Rey, JW. (2021).
  
  \smallskip
  \href{https://www.wjgnet.com/1948-5190/full/v13/i7/221.htm}
    {Endoscopic hemostasis makes the difference: Angiographic treatment in
    patients with lower gastrointestinal bleeding}

  \smallskip
  \textit{World J Gastrointest Endosc, 13(7): 221-232}
\end{tcolorbox}

% ┌────────────────────────────────────────────────────────────────────────────┐
% │ NANOSTRING PAPER                                                           │
% │   Advance of theragnosis biomarkers in lung cancer: from clinical to       │
% │   molecular pathology and biology                                          │
% └────────────────────────────────────────────────────────────────────────────┘

\chapter{Theragnosis Biomarkers in Lung Cancer}

\section{Project Introduction}

\section{Failed Explorations}

\begin{tcolorbox}[
  boxrule=0pt, leftrule=1pt, colframe=s-blue, colback=white, sharp corners=all]%
  \raggedright
  Alidousty, C., Baar, T., Heydt, C., Wagener-Ryczek, S., Kron, A., Wolf, J.,
  Buettner, R., Schultheis, AM. (2018).
  
  \smallskip
  \href{http://jtd.amegroups.com/article/view/25843}
    {Advance of theragnosis biomarkers in lung cancer: from clinical to
    molecular pathology and biology}

  \smallskip
  \textit{J Thorac Dis, 11(1):3-8.}
\end{tcolorbox}

% ┌────────────────────────────────────────────────────────────────────────────┐
% │ CONCLUSION                                                                 │
% └────────────────────────────────────────────────────────────────────────────┘

\chapter{Conclusion}

% ┌────────────────────────────────────────────────────────────────────────────┐
% │ END                                                                        │
% └────────────────────────────────────────────────────────────────────────────┘

\chapter{Introduction}
An amalgamation of technological advancement and sophisticated magic offers many 
amenities to the people of Ravnica (amenities that would be extraordinary to
folk in more common fantasy settings). Almost all residential areas of the city
enjoy central heating and plumbing, thanks to the work of the Izzet League, as
well as elevators, and spacious apartments. There also exists an extensive
public transport system, making every part of the city easily reachable.
\begin{align*}
t_{\nicefrac{1}{2}}=\frac{\ln\left(2\right)}{\delta}
\end{align*}

\section{Section}
Even poorer neighbourhoods boast clean and smooth roads and sturdy construction.
No one needs to go hungry in Ravnica, because the Golgari Union provides a bare
minimum of  sustenance to anyone who can't afford better food, though it is best
not to think too much about where the thick gruel comes from. The citizens of
Ravnica enjoy plenty of leisure time, as full-time employment in most cases
means seven working hours per day, and the city offers an abundance of ways to
fill it. Ravnica features restaurants with wide-ranging collections of fine
wines, cafés serving coffee and tea, street vendors offering portable meals, and
bakeries that sell a wide variety of breads and pastries \citep{Reich2013}.

Travellers can stay in luxury hotels or simple hostels, or they can rely on
their personal or guild-related contacts to find housing. \marginnote{This is a
really long margin note that hopefully extends correctly over multiple lines.}
Diversions and entertainments abound, including raucous street-side theatre,
operas and symphonies, illegal fight clubs, sporting events held in vast
arenas, throwaway popular novels, and great works of literature.

\subsection{Subsection}
These things are shared by the city's diverse peoples, who enjoy a life adorned
by a variety of  species, gender identities, and sexual orientations.
Well-established systems undergird society, largely through the efforts of the
guilds. The Azorius Senate crafts and codifies a comprehensive, though some
would say oppressive, set of laws, enforced in turn by the Boros Corps. The Cult
of Rakdos cares for the simple working class and criminal underworld, and the
Golgari Union ensures that waste is disposed of and recycled. The Izzet League
maintains the city's infrastructure, and the Orzhov Syndicate provides secure
vaults and intricate financial arrangements, while their lawyers offer their
services to anyone who can afford them. The Selesnya Conclave addresses issues
of education and public health, as well as local public transport, while the
Simic Combine sees to the biological maintenance and improvement of all
citizens. \marginnote{This is a note}

\subsubsection{Subsubsection}
Few parts of Ravnica could be considered wilderness. There exist abandoned sites
where the infrastructure lies in ruins, partially reclaimed by natural forces.
These are the only truly wild areas of the city, but even those do not last
forever. It may take a hundred years or more, but sooner or later, these areas
are reclaimed by civilization. In the meantime tough, they serve as safe havens
for fugitives, malcontents, and failed, unlicensed experiments in
thaumatological bioengineering that conveniently eloped from a Simic Combine
research enclave.

These things are shared by the city's diverse peoples, who enjoy a life adorned
by a variety of  species, gender identities, and sexual orientations.
Well-established systems undergird society, largely through the efforts of the
guilds. The Azorius Senate crafts and codifies a comprehensive, though some
would say oppressive, set of laws, enforced in turn by the Boros Corps. The Cult
of Rakdos cares for the simple working class and criminal underworld, and the
Golgari Union ensures that waste is disposed of and recycled. The Izzet League
maintains the city's infrastructure, and the Orzhov Syndicate provides secure
vaults and intricate financial arrangements, while their lawyers offer their
services to anyone who can afford them. The Selesnya Conclave addresses issues
of education and public health, as well as local public transport, while the
Simic Combine sees to the biological maintenance and improvement of all
citizens.

These things are shared by the city's diverse peoples, who enjoy a life adorned
by a variety of  species, gender identities, and sexual orientations.
Well-established systems undergird society, largely through the efforts of the
guilds. The Azorius Senate crafts and codifies a comprehensive, though some
would say oppressive, set of laws, enforced in turn by the Boros Corps. The Cult
of Rakdos cares for the simple working class and criminal underworld, and the
Golgari Union ensures that waste is disposed of and recycled. The Izzet League
maintains the city's infrastructure, and the Orzhov Syndicate provides secure
vaults and intricate financial arrangements, while their lawyers offer their
services to anyone who can afford them. The Selesnya Conclave addresses issues
of education and public health, as well as local public transport, while the
Simic Combine sees to the biological maintenance and improvement of all
citizens.

\chapter{Methods}
These things are shared by the city's diverse peoples, who enjoy a life adorned
by a variety of  species, gender identities, and sexual orientations.
Well-established systems undergird society, largely through the efforts of the
guilds. The Azorius Senate crafts and codifies a comprehensive, though some
would say oppressive, set of laws, enforced in turn by the Boros Corps. The Cult
of Rakdos cares for the simple working class and criminal underworld, and the
Golgari Union ensures that waste is disposed of and recycled. The Izzet League
maintains the city's infrastructure, and the Orzhov Syndicate provides secure
vaults and intricate financial arrangements, while their lawyers offer their
services to anyone who can afford them. The Selesnya Conclave addresses issues
of education and public health, as well as local public transport, while the
Simic Combine sees to the biological maintenance and improvement of all
citizens. \marginnote{text}

These things are shared by the city's diverse peoples, who enjoy a life adorned
by a variety of  species, gender identities, and sexual orientations.
Well-established systems undergird society, largely through the efforts of the
guilds. The Azorius Senate crafts and codifies a comprehensive, though some
would say oppressive, set of laws, enforced in turn by the Boros Corps. The Cult
of Rakdos cares for the simple working class and criminal underworld, and the
Golgari Union ensures that waste is disposed of and recycled. The Izzet League
maintains the city's infrastructure, and the Orzhov Syndicate provides secure
vaults and intricate financial arrangements, while their lawyers offer their
services to anyone who can afford them. The Selesnya Conclave addresses issues
of education and public health, as well as local public transport, while the
Simic Combine sees to the biological maintenance and improvement of all
citizens.

\begin{table}[t!]
\caption[Currently available pathogen detection methods.]{This table lists six
different currently available products to which the developed approach has been
compared. Hybcell (Anagnostics) is a microarray-based diagnostic method which
uses molecular markers to identify bacterial and fungal pathogens, and can also
asses antibiotic resistance.}
\begin{tabularx}{\textwidth}{llXcr}
  Product   &             & Method     & Pathogens & Publication             \\
  \midrule
  Hybcell   & Anagnostics & microarray &           &                         \\
  Prove-it  & Mobidiag    & microarray & 73        & \citealp{Aittakorpi2012}\\
  SeptiFast & Roche       & PCR        & 25        & \citealp{Vince2008}     \\
  SepsiTest & Molzym      & PCR        &           &                         \\
  Unyvero   & Curetis     & microarray & 35        & \citealp{Jamal2014}     \\
  VITEK     & bioMerieux  & culture    &           &                         \\
  \bottomrule
\end{tabularx}
\label{tab:competitors}
\end{table}

These things are shared by the city's diverse peoples, who enjoy a life adorned
by a variety of  species, gender identities, and sexual orientations.
Well-established systems undergird society, largely through the efforts of the
guilds. The Azorius Senate crafts and codifies a comprehensive, though some
would say oppressive, set of laws, enforced in turn by the Boros Corps. The Cult
of Rakdos cares for the simple working class and criminal underworld, and the
Golgari Union ensures that waste is disposed of and recycled. The Izzet League
maintains the city's infrastructure, and the Orzhov Syndicate provides secure
vaults and intricate financial arrangements, while their lawyers offer their
services to anyone who can afford them. The Selesnya Conclave addresses issues
of education and public health, as well as local public transport, while the
Simic Combine sees to the biological maintenance and improvement of all
citizens.

These things are shared by the city's diverse peoples, who enjoy a life adorned
by a variety of  species, gender identities, and sexual orientations.
Well-established systems undergird society, largely through the efforts of the
guilds. The Azorius Senate crafts and codifies a comprehensive, though some
would say oppressive, set of laws, enforced in turn by the Boros Corps. The Cult
of Rakdos cares for the simple working class and criminal underworld, and the
Golgari Union ensures that waste is disposed of and recycled. The Izzet League
maintains the city's infrastructure, and the Orzhov Syndicate provides secure
vaults and intricate financial arrangements, while their lawyers offer their
services to anyone who can afford them. The Selesnya Conclave addresses issues
of education and public health, as well as local public transport, while the
Simic Combine sees to the biological maintenance and improvement of all
citizens.

These things are shared by the city's diverse peoples, who enjoy a life adorned
by a variety of  species, gender identities, and sexual orientations
\citep{Trapnell2010, Scholz2003, Schweitzer2011}. Well-established systems undergird society, largely
through the efforts of the guilds. The Azorius Senate crafts and codifies a
comprehensive, though some would say oppressive, set of laws, enforced in turn
by the Boros Corps. The Cult of Rakdos cares for the simple working class and
criminal underworld, and the Golgari Union ensures that waste is disposed of and
recycled. The Izzet League maintains the city's infrastructure, and the Orzhov
Syndicate provides secure vaults and intricate financial arrangements, while
their lawyers offer their services to anyone who can afford them. The Selesnya
Conclave addresses issues of education and public health, as well as local
public transport, while the Simic Combine sees to the biological maintenance and
improvement of all citizens.

These things are shared by the city's diverse peoples, who enjoy a life adorned
by a variety of  species, gender identities, and sexual orientations.
Well-established systems undergird society, largely through the efforts of the
guilds. The Azorius Senate crafts and codifies a comprehensive, though some
would say oppressive, set of laws, enforced in turn by the Boros Corps. The Cult
of Rakdos cares for the simple working class and criminal underworld, and the
Golgari Union ensures that waste is disposed of and recycled. The Izzet League
maintains the city's infrastructure, and the Orzhov Syndicate provides secure
vaults and intricate financial arrangements, while their lawyers offer their
services to anyone who can afford them. The Selesnya Conclave addresses issues
of education and public health, as well as local public transport, while the
Simic Combine sees to the biological maintenance and improvement of all
citizens.

\section{Half-Life Estimation}

Explanation of how we come to model the degradation rate
$\delta$\marginnote{Usually, degradation rate is denoted by $\lambda$ not
$\delta$, but the Poisson distribution's parameter is $\lambda$ already.} of Alu
element transcripts. In our experiment, we measured Alu element expression using
DTA (dynamic transcriptome analysis) with 4sU labelling. In this protocol, newly
created transcripts are labelled, making them distinguishable from transcripts
created before the labelling pulse. In the end, we want to obtain an estimate of
the half-life $t_{\nicefrac{1}{2}}$ of each individual Alu element, which can be
calculate from the degradation rate
$\delta$ by
\begin{equation*}
t_{\nicefrac{1}{2}}=\frac{\ln\left(2\right)}{\delta}
\end{equation*}

\noindent The labelling pulse's duration $\Delta t$ is 5 min, meaning that 5 min
passed after the labelling agent was added and before the amount of labelled
transcripts within the cell was measured.

With time, new transcripts are created and old transcript are degraded. This
means that after the labelling pulse, the amount of unlabelled transcripts
continually decreases, because all transcripts created after the labelling pulse
are labelled, until only labelled transcripts remain. We assume that transcript
degradation follows exponential decay, while the total amounts of transcripts
in a cell remains constant, as transcription and transcript decay are in
equilibrium. Therefore, the total amount of transcripts remains constant, but
the ratio of labelled reads increases exponentially.

\bigbreak
\noindent\begin{minipage}{\textwidth}
\centering
% Created by tikzDevice version 0.12.3.1 on 2020-07-26 17:02:37
% !TEX encoding = UTF-8 Unicode
\begin{tikzpicture}[x=1pt,y=1pt]
\definecolor{fillColor}{RGB}{255,255,255}
\path[use as bounding box,fill=fillColor,fill opacity=0.00] (0,0) rectangle (180.67, 94.39);
\begin{scope}
\path[clip] ( 16.57,  0.00) rectangle (164.10, 94.39);
\definecolor{drawColor}{RGB}{255,255,255}
\definecolor{fillColor}{RGB}{255,255,255}

\path[draw=drawColor,line width= 0.5pt,line join=round,line cap=round,fill=fillColor] ( 16.57, -0.00) rectangle (164.10, 94.39);
\end{scope}
\begin{scope}
\path[clip] ( 41.40, 27.90) rectangle (159.10, 89.39);
\definecolor{fillColor}{RGB}{255,255,255}

\path[fill=fillColor] ( 41.40, 27.90) rectangle (159.10, 89.39);
\definecolor{drawColor}{RGB}{0,0,0}

\path[draw=drawColor,line width= 0.6pt,line join=round] ( 46.75, 30.69) --
	( 47.82, 33.44) --
	( 48.89, 36.05) --
	( 49.96, 38.53) --
	( 51.03, 40.89) --
	( 52.10, 43.14) --
	( 53.17, 45.28) --
	( 54.24, 47.31) --
	( 55.31, 49.25) --
	( 56.38, 51.09) --
	( 57.45, 52.84) --
	( 58.52, 54.50) --
	( 59.59, 56.08) --
	( 60.66, 57.59) --
	( 61.73, 59.02) --
	( 62.80, 60.39) --
	( 63.87, 61.68) --
	( 64.94, 62.92) --
	( 66.01, 64.09) --
	( 67.08, 65.20) --
	( 68.15, 66.27) --
	( 69.22, 67.28) --
	( 70.29, 68.24) --
	( 71.36, 69.15) --
	( 72.43, 70.02) --
	( 73.50, 70.85) --
	( 74.57, 71.63) --
	( 75.64, 72.38) --
	( 76.71, 73.09) --
	( 77.78, 73.77) --
	( 78.85, 74.41) --
	( 79.92, 75.03) --
	( 80.99, 75.61) --
	( 82.06, 76.16) --
	( 83.13, 76.69) --
	( 84.20, 77.19) --
	( 85.27, 77.67) --
	( 86.34, 78.12) --
	( 87.41, 78.55) --
	( 88.48, 78.96) --
	( 89.55, 79.35) --
	( 90.62, 79.73) --
	( 91.69, 80.08) --
	( 92.76, 80.41) --
	( 93.83, 80.73) --
	( 94.90, 81.04) --
	( 95.97, 81.33) --
	( 97.04, 81.60) --
	( 98.11, 81.86) --
	( 99.18, 82.11) --
	(100.25, 82.35) --
	(101.32, 82.58) --
	(102.39, 82.79) --
	(103.46, 82.99) --
	(104.53, 83.19) --
	(105.60, 83.37) --
	(106.67, 83.55) --
	(107.74, 83.71) --
	(108.81, 83.87) --
	(109.88, 84.02) --
	(110.95, 84.17) --
	(112.02, 84.30) --
	(113.09, 84.43) --
	(114.16, 84.56) --
	(115.23, 84.68) --
	(116.30, 84.79) --
	(117.37, 84.89) --
	(118.44, 85.00) --
	(119.51, 85.09) --
	(120.58, 85.18) --
	(121.65, 85.27) --
	(122.72, 85.35) --
	(123.79, 85.43) --
	(124.86, 85.51) --
	(125.93, 85.58) --
	(127.00, 85.65) --
	(128.07, 85.71) --
	(129.14, 85.77) --
	(130.21, 85.83) --
	(131.28, 85.89) --
	(132.35, 85.94) --
	(133.42, 85.99) --
	(134.49, 86.04) --
	(135.56, 86.08) --
	(136.63, 86.13) --
	(137.70, 86.17) --
	(138.77, 86.21) --
	(139.84, 86.24) --
	(140.91, 86.28) --
	(141.98, 86.31) --
	(143.05, 86.35) --
	(144.12, 86.38) --
	(145.19, 86.40) --
	(146.26, 86.43) --
	(147.33, 86.46) --
	(148.40, 86.48) --
	(149.47, 86.51) --
	(150.54, 86.53) --
	(151.61, 86.55) --
	(152.68, 86.57) --
	(153.75, 86.59);

\path[draw=drawColor,line width= 0.5pt,line join=round,line cap=round] ( 41.40, 27.90) rectangle (159.10, 89.39);
\end{scope}
\begin{scope}
\path[clip] (  0.00,  0.00) rectangle (180.67, 94.39);
\definecolor{drawColor}{RGB}{0,0,0}

\node[text=drawColor,anchor=base east,inner sep=0pt, outer sep=0pt, scale=  0.80] at ( 36.90, 27.94) {0};

\node[text=drawColor,anchor=base east,inner sep=0pt, outer sep=0pt, scale=  0.80] at ( 36.90, 84.22) {1};
\end{scope}
\begin{scope}
\path[clip] (  0.00,  0.00) rectangle (180.67, 94.39);
\definecolor{drawColor}{RGB}{0,0,0}

\path[draw=drawColor,line width= 0.3pt,line join=round] ( 38.90, 30.69) --
	( 41.40, 30.69);

\path[draw=drawColor,line width= 0.3pt,line join=round] ( 38.90, 86.97) --
	( 41.40, 86.97);
\end{scope}
\begin{scope}
\path[clip] (  0.00,  0.00) rectangle (180.67, 94.39);
\definecolor{drawColor}{RGB}{0,0,0}

\path[draw=drawColor,line width= 0.3pt,line join=round] ( 46.75, 25.40) --
	( 46.75, 27.90);

\path[draw=drawColor,line width= 0.3pt,line join=round] (153.75, 25.40) --
	(153.75, 27.90);
\end{scope}
\begin{scope}
\path[clip] (  0.00,  0.00) rectangle (180.67, 94.39);
\definecolor{drawColor}{RGB}{0,0,0}

\node[text=drawColor,anchor=base,inner sep=0pt, outer sep=0pt, scale=  0.80] at ( 46.75, 17.89) {0};

\node[text=drawColor,anchor=base,inner sep=0pt, outer sep=0pt, scale=  0.80] at (153.75, 17.89) {t};
\end{scope}
\begin{scope}
\path[clip] (  0.00,  0.00) rectangle (180.67, 94.39);
\definecolor{drawColor}{RGB}{0,0,0}

\node[text=drawColor,anchor=base,inner sep=0pt, outer sep=0pt, scale=  1.00] at (100.25,  6.94) {Time};
\end{scope}
\begin{scope}
\path[clip] (  0.00,  0.00) rectangle (180.67, 94.39);
\definecolor{drawColor}{RGB}{0,0,0}

\node[text=drawColor,rotate= 90.00,anchor=base,inner sep=0pt, outer sep=0pt, scale=  1.00] at ( 28.46, 58.64) {L / T};
\end{scope}
\end{tikzpicture}

\vspace{-\ht\strutbox}
\captionof{figure}[Exponential RNA degradation]{Model illustration of the
exponential RNA degradation and the resulting $L/T$ ratio over time.}
\label{fig:expdecay}
\end{minipage}
\bigbreak

\noindent Let $t_{a}$ be the total number of molecules of any Alu element $a$ in
solution. Among those $t_{a}$ molecules, let $l_{a}$ denote the number of newly
synthesized (and therefore labelled) molecules. By assumption of steady state
conditions and an exponential decay, the ratio between labelled and total
molecules $r_{a}$ for any Alu element $a$ is given by
\begin{align*}
  r_{a}     &= \nicefrac{l_{a}}{t_{a}}=1-\exp(-\delta_{g}\Delta t)
\\ \ln r_{a} &= \ln\left(1-\exp(-\delta_{g}\Delta t)\right)
\end{align*}

\noindent where $\delta_{a}$ is the degradation rate of any Alu element $a$. The
total amount of molecules $t_{a}$ of any Alu element $a$ is also given as:
\begin{equation*}
t_{a}=\nicefrac{\mu_{a}}{\delta_{a}}
\end{equation*}

\noindent where $\mu_{a}$ is the synthesis rate of any Alu element $a$.

\noindent\marginnote{This approximation could be replaced by the solution of the
standard ODE for RNA metabolism, or by a term that takes into account the
non-constant labelling efficiency for short labelling periods.}Neglecting the
decay of newly synthesized transcripts, we likewise obtain
\begin{equation*}
l_{a}=\mu_{a}\Delta t
\end{equation*}

\noindent We further assume that
\begin{equation*}
l_{all}=\sum_{a}l_{a}\quad\text{and}\quad t_{all}=\sum_{a}t_{a}
\end{equation*}

\noindent We now shift our view from the molecules in solution to read counts
obtained through sequencing. We prepare a sequencing library with $N_{tot}$
reads. After 4sU pull-down, we prepare a sequencing library of size $N_{lab}$
reads. The distribution of total counts $T_{a}$ and labelled counts $L_{a}$
respectively for any Alu element $a$ is then\marginnote{There is also the
possibility to choose a (zero-inflated) negative binomial distribution instead.}
\begin{align*}
  T_{a} &\sim \operatorname{Pois}
  (\lambda_{tot}=\frac{t_{a}}{t_{all}}\cdot N_{tot})
\\ L_{a} &\sim \operatorname{Pois}
  (\lambda_{lab}=\frac{l_{a}}{l_{all}}\cdot N_{lab})
\end{align*}

\noindent We thus assume that the estimated ratio between labelled and total
molecules $\hat{r}_{a}$ for any Alu element $a$ is given by
\begin{equation*}
\hat{r}_{a}=\frac{L_{a}}{T_{a}}\cdot\frac{N_{tot}}{N_{lab}}
\end{equation*}

\noindent We still need to estimate $\nicefrac{l_{all}}{t_{all}}$ i.e., the
fraction of (all) labelled molecules among all molecules in solution. We use
spike-ins to do so. Let $l_{\text{spk}}$ be the number of labelled spike-in
molecules, and $t_{\text{spk}}$ the number of total spike-in molecules that were
added to the solution. We know that
\begin{equation}
\frac{l_{spk}}{l_{all}}\approx\frac{L_{spk}}{L_{all}}=\frac{L_{spk}}{N_{lab}}
\quad\text{and}\quad
\frac{t_{spk}}{t_{all}}\approx\frac{T_{spk}}{T_{all}}=\frac{T_{spk}}{N_{tot}}
\label{spk}
\end{equation}

\noindent The term $q=\nicefrac{l_{spk}}{t_{spk}}$ is thus known by composition
of the spike-in reagents.