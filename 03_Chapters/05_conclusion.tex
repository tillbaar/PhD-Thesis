% ┌────────────────────────────────────────────────────────────────────────────┐
% │ CONCLUSION                                                                 │
% └────────────────────────────────────────────────────────────────────────────┘

\chapter{Conclusion}
\sectionmark{Conclusion}

As the three included publications demonstrate, the methods to analyse
biological data can be as varied as the data itself. In the following, some
more general insights gained after the conclusion of the individual projects
are discussed.

\subsubsection{Evolution Shapes the Alu RNA Metabolism}
\label{subsubsec:alus}
Of the three included publications, this project is the methodologically most
complex one, as it essentially addresses four separate questions based on the
same RNA-seq data. This shows the breadth of possible approaches that can be
taken when analysing biological data. The data was used to investigate general
expression patterns, differential expression under Pol-II inhibition, and
sequence features, too.

We also looked into different ways to answer whether Alu elements are
transcribed by Pol-II or Pol-III, but these did not result in conclusive
answers. We examined chromatin immunoprecipitation sequencing (ChIPseq),
trying to correlate Pol-II and Pol-III peaks with Alu expression, but the
resolution of the data we could obtain was not enough to draw any solid
conclusions. We also tried to exploit the 5\textquotesingle-cap structure to
differentiate transcripts, but this would have required a different
experimental setup and did not appear promising in the first place. Finally,
we also considered using genomic run-on sequencing (GROseq), which limits the
sequencing to nascent RNAs that are currently transcribed by a polymerase, but
at the time of data collection, the method was not developed enough to
generate both Pol-II and Pol-III data of sufficient quality \citep{Gardini2017}.

To continue the investigation into the life cycle of Alu elements, the main
point left unanswered is the individual transcript- or loci-level origin of
Alu RNAs. Three strategies present themselves but would require new
experiments. Firstly, synthetic inhibitors specific to Pol-II or Pol-III have
been developed that work more efficiently than \textalpha-amanitin. These
could be used in conjunction with deeper RNA-seq to perform a high-resolution
differential expression analysis with less biological noise caused by the long
incubation time required by \textalpha-amanitin. Secondly, GRO-seq could be
used to ascertain the origin of individual Alu transcripts. Thirdly, an
\textit{in vitro} experiment could be used, combining selected Alu DNA
fragments with either Pol-II or Pol-III, observing which of the fragments are
targeted by which polymerase.

\subsubsection{Angiography for Gastrointestinal Bleeding}
\label{subsubsec:angio}

This project demonstrates that the goal is not always to choose the method
that can model the data most accurately. During the course of the
investigation, we discussed several potential candidates, such as GLMs and
random forests (see \nameref{sec:methoverview}). In retrospect, these
techniques could have resulted in a model closer to the optimal balance
between bias and variance (see \nameref{chap:introduction}). However, these
techniques would also have lacked the interpretability offered by a simple
decision tree.

As our goal was to construct a decision-making aid, easy to use in the daily
clinic routine, it turned out that, in the end, slight losses in model
accuracy and specificity were acceptable trade-offs for improved practicability.

Another
%\marginnote{Born-Again Trees}\label{mar:batrees} 
method we could have applied that might have resulted in a similarly
interpretable model are born-again tree ensembles \citep{Sagi2020,Vidal2020}.
This method is proven to transform a random forest model back into a single
minimal-size decision tree, the born-again (BA) tree, with the optimal number
of leaves and a faithful feature space representation. The underlying
algorithm was tested on different data sets, including medical data. However,
a high number of features can lead to severe decreases in performance, and an
upper limit of \num{20} features is recommended by the authors. Also, the BA
tree can still be very complex with over \num{1000} leaves, which would again
make the method impractical for the daily clinic routine. While pruning of BA
trees is implemented to simplify the final result, the accuracy of the final
BA tree is then no longer guaranteed. Still, testing showed only negligible
losses in accuracy. Should this project be continued, possible with a much
larger data set, BA trees could be a good model choice.

Finally, this project also showed the importance of data cleaning and
validation, and how essential it can be to set up analyses in a reproducible
manner. Throughout the investigation, the data set had to be amended and
corrected numerous times, as each new descriptive report revealed new
inconsistencies and errors in the original records, which is not surprising
considering that the data was collected over a ten-year period. Would the
analyses have needed to be re-run manually, surely this publication would have
spend a few more months in preparation.

\subsubsection{Theragnosis Biomarkers in Lung Cancer}
\label{subsubsec:alk}

This investigation exemplifies how important the choice of the best-suited
measurement technology is. Would we not have used the NanoString nCounter
method (see \cpageref{mar:nanostring}) but more conventional bulk RNA-seq, the
analysis would most likely have been much more complicated. Copy number
variant detection from RNA-seq was, at least at the time of the publication of
our study, still very unreliable and largely impossible with panel-based
sequencing, which is common in the clinical environment. Since then, advances
have been made, and tools like CaSpER could present an alternative route to
take in future investigations \citep{SerinHarmanci2020}.
\bigbreak

\noindent 
In conclusion, this dissertation shows that the analysis of high-dimensional,
biological data, and regression, in particular, is a broad field with many
forks in the road. While first and foremost an exact scientific discipline, of
course, choosing the proper method to answer a research question is also an art.