% ┌────────────────────────────────────────────────────────────────────────────┐
% │ INTRODUCTION                                                               │
% └────────────────────────────────────────────────────────────────────────────┘

\chapter{Introduction}\label{chap:introduction}
\sectionmark{Introduction} % This sets the outer header mark to "Introduction"
%                          % until the first section is begun.

\enquote{The temptation to form premature theories upon insufficient data is 
the bane of our profession.} [Sherlock Holmes, \citealt{Holmes}]. This 
statement applies not only to fictional consulting detectives but likewise to 
the field of science. One might even say that it is an inherent tendency of 
the human mind to jump to conclusions based on incomplete knowledge. Thus, it
is the statistician's task to rigidify all conjectures made in the context of
a scientific investigation and base them firmly on observations. This is, of
course, easier said than done.

In general, the art of the statistician, in contrast to purely technical 
aspects, is to specify the model according to the data at hand. Upon contact 
with data, every mathematical model suffers from the bias-variance tradeoff
\citep{VonLuxburg2009}, i.e., a model needs to find the right balance between 
overfitting (high variance) and underfitting (high bias). Bias, in this case, 
describes the difference between a model's average predictions and the true 
values. A model with high bias is oversimplified. Variance refers to the 
variability of a model's predictions given the true values. A model with high 
variance fails to generalise and cannot make accurate predictions.
\begin{align*}
  \intertext{Following an example by \citealt{James2009}, assume we wish to
  predict an outcome $Y$ given observations $X=\left\{x_1\dots x_n\right\}$
  with a relationship of}
  Y =&\, f\!\left(X\right)+\varepsilon
  \intertext{where $\varepsilon$ describes a normally distributed error term
  with mean 0. Further assuming a model $\widehat{f}\!\left(X\right)$ of
  $f\!\left(X\right)$, the expected squared error for a point $x$ becomes}
  \operatorname{E}\left(x\right) =&\,
    \operatorname{E}\left[\left(Y-\widehat{f}\!\left(x\right)\right)^2\right]
  \intertext{with $\operatorname{E}$ denoting the expected value. This further
  decomposes to}
  \operatorname{E}\left(x\right) =& 
     \underbracket{\vphantom{\left[\left(\widehat{f}\right)^2\right]}
     \left(\operatorname{E}\left[\widehat{f}\!\left(x\right)\right]-
      f\!\left(x\right)\right)^2}_{\text{Bias}^2}+
      \underbracket{\operatorname{E}\left[\left(\widehat{f}\!\left(x\right)-
      \operatorname{E}\left[\widehat{f}\!\left(x\right)\right]\right)^2\right]}
      _{\text{Variance}}+
      \,\sigma^2_\varepsilon
  \shortintertext{where $\sigma^2_\varepsilon$ is the irreducible error,
  i.e., the amount of inherent noise in the data that cannot be removed.}
\end{align*}
% To reduce the vertical spacing between the align environment and the
% following text, use \shortintertext instead of \intertext for the last
% paragraph and follow the align environment with the command:
% \vspace{-\the\belowdisplayskip} to remove unnecessary space.
% ──────────────────────────────────────────────────────────────────────
% Ugly, I know, but it works and doesn't require too much effort.

\vspace{-\the\belowdisplayskip}
\noindent In other words, a model that does not capture the pattern from which 
the observations emerge is underfitted, usually exhibiting high bias and low 
variance. Vice versa, a model that captures the noisiness of the observations
alongside the pattern is overfitted, usually exhibiting low bias and high
variance.

The bias-variance tradeoff is connected to the complexity of a model. A model 
that is too simple and thus undercomplex for the data will underfit. This is 
because it has too few parameters to model the data adequately. Conversely, a 
model that is overcomplex will overfit, as it has too many parameters. Of
course, one would hope to optimally balance each model's complexity, bias, and
variance so to never over- or underfit, or at least come feasible close to
this goal.

To approach the optimal model, commonly used methods are regularisation,
boosting, and bagging.
\bigbreak

\noindent Regularisation \marginnote{Regularisation}\label{mar:regularisation} 
aims to mitigate the problem of overfitting common to overcomplex models
\citep {Deisenroth2020}. Assume again a model that predicts $f\!\left(X\right)$
and possesses many parameters $\theta_1\dots\theta_n$. Its associated loss 
function $V$ governs the training of the model \citep{Rosasco2003}. 
Regularisation adds a regularisation term or regulariser $R$ to the loss 
function that penalises complexity of the model. Thus, the expression to be 
minimised becomes
\begin{align*}
  \min_{\widehat{f}}\sum_{i=1}^n
  V\!\left(\widehat{f}\!\left(x_i\right),f\!\left(x_i\right)\right)+
  \lambda\:R\!\left(\widehat{f}\:\right)
\end{align*}
with the parameter $\lambda$ controlling the amount of regularisation that is
applied.

Two common applications of regularisation are the linear regression techniques 
Ridge regression \citep{Hoerl1970} and Lasso regression \citep{Tibshirani1996}.
While these can be extended to other statistical models, assume simple linear
regression for the sake of demonstration. Both techniques add a regularisation
term to the loss function that depends directly on the values of the model's
parameters $\theta_i$
\begin{align*}
  \underset{\text{Ridge regression}}{
    R\left(\theta_{i}\right)=\lambda\sum_{i=1}^{n}\theta_{\,i}^{\,2}
  }
  \quad\text{and}\quad
  \underset{\text{Lasso regression}}{
    R\left(\theta_{i}\right)=\lambda\sum_{i=1}^{n}\left|\,\theta_{\,i}\right|
  }
\end{align*}
thus shrinking all but the most influential parameters of the model and 
thereby reducing model complexity and multicollinearity \citep{Herawati2018}. 
The difference between Ridge and Lasso regression lies in the calculation of 
the applied penalty. While Ridge regression penalises the sum of the squared 
coefficients (L2 penalty), Lasso regression penalises the sum of their 
absolute values (L1 penalty). The ultimate consequence is that while Lasso can 
shrink non-influential parameters to zero, Ridge cannot. On the other hand, 
this can cause Lasso to eliminate important parameters under 
multicollinearity, if predictor variables are correlated, as it tends to 
select one parameter from the correlated group and ignore the rest.

To overcome these limitations, a combination of Ridge and Lasso regression can 
be applied, elastic net \citep{Zou2005}. The used regularisation technique 
combines an L1 and an L2 penalty by using separate $\lambda$ parameters for 
each, $\lambda_{1}$ and $\lambda_{2}$. If $\lambda_{1}=0$, the penalty equals 
Ridge regularisation; if $\lambda_{2}=0$, the penalty equals Lasso 
regularisation; and if $\lambda_{1}>0$ and $\lambda_{2}>0$, a combination of 
both is applied.
\bigbreak

\noindent While \marginnote{Boosting}\label{mar:boosting} regularisation is a 
helpful method to deal with overcomplex models, boosting addresses the problem 
of poor models in more general terms \citep{Freund1999}; a \tqs{poor model}, 
in this case, refers to a weak learner. \citet{ Valiant} formalised the 
concept of learnability in the context of computational complexity theory and 
introduced the probably approximately correct (PAC) model. A problem is PAC-
learnable if there exists a model that, with a chance higher than a threshold
$\delta$, will arrive at a solution with a generalisation error smaller than a 
threshold $\epsilon$. Generalisation error or out-of-sample error refers to a 
model's predictive performance on previously unseen data \citep{Bousquet2004}. 
A model satisfying these conditions for any given problem is called a strong 
learner, while a model that does not is a weak learner.

A problem can benefit from boosting if applying a strong learner is either 
impossible, as no strong learner exists, or disadvantageous, for example, 
because the strong learner is prohibitively complex and thus underperformant, 
or because the available training data is insufficient to apply it. While 
current machine learning research, especially in the field of deep learning 
\citep{LeCun2015}, mainly approaches the challenge of more complex problems by 
fielding stronger algorithms, boosting seeks to improve the results of weak 
learners.

While \citet{Kearns} defined weak learners as models that perform just 
slightly better than random guessing, \citet{Schapire1990} demonstrated their 
power if applied correctly, proving that any problem solvable by a strong 
learner is equally solvable by a collection of weak learners: the hypothesis 
boosting mechanism. The term \tqs{hypothesis} here describes the solution a 
model arrives at after training, the model's final parameters. \citet{
Freund1995} improved this further, combining many weak learners and using 
their combined results to arrive at a strong prediction, one weak learner 
effectively compensating for the shortcomings of another. The next step was 
AdaBoost, adaptive boosting, for which Freund and Schapire were awarded the 
Gödel Prize in 2003 \citep{Freund1997}. This boosting variant scales each weak 
learner's influence on the final prediction depending on their own error. The 
current state-of-the-art boosting technique is gradient boosting with its 
predominant implementation XGBoost \citep{Chen2016}. In contrast to AdaBoost, 
which always minimises the exponential loss function, gradient boosting can 
use any differentiable loss function, which makes it adaptable to many 
classification and regression tasks.

Following an example by \citet{Li}, assume again a model
$\widehat{f}\!\left(X\right)$. The boosting model iterates over $M$ stages, 
and each stage $m$ has an associated imperfect model
$\widehat{f}\!\left(X\right)_{m}$, so that at each stage, a new
\tqs{hypothesis} is added, a new estimator
$\widehat{g}\left(X\right)_{m}$\vphantom{$\widehat{f}$}.
\begin{align*}
  \widehat{f}\!\left(X\right)_{m+1}&=\widehat{f}\!\left(X\right)_{m}+
    \widehat{g}\left(X\right)_{m}
  \intertext{As the model iterates over sets of training data
  $Y_{i}=\left\{y_1\dots y_n\right\}$, the new estimator is fit to the
  residual, the difference between the values of the training data $Y_{i}$ and
  the estimation of the previous model.}
  \widehat{g}\left(X\right)_{m}&=Y_{i}-\widehat{f}\!\left(X\right)_{m}
\end{align*}
In this fashion, each new stage $m+1$ attempts to correct for the errors made
 by the previous stage $m$.
\bigbreak

\noindent While \marginnote{Bagging}\label{mar:bagging} regularisation is an
ensemble method that addresses shortcomings of the model, bagging can be
regarded as addressing shortcomings of the data \citep{Breiman1996}. The name
\tqs{bagging} derives from bootstrap aggregating. Bootstrapping is a
resampling method that estimates statistics by sampling repeatedly from the
same data \citep{Efron1994}. This process makes it possible to assess the
accuracy of the estimated statistics, which can be assumed to be an adequate
approximation if the empirical distribution of the data represents the true
distribution reasonably well.

Bagging applies the principle of bootstrapping to model training. Following an
example by \citet{Aslam2007}, assume again a set of training data $Y\!$. From
this data set, new training sets $Y_{i}$ are created by sampling uniformly
from $Y\!$ with replacement, meaning that each individual entry in $Y\!$ has
the same probability of being drawn and can be drawn again and again.
Individual models are then trained on the new training sets $Y_{i}$, and their
predictions are combined, either by averaging for regression or voting for
classification (see \nameref{subsubsec:regclas} below). Bagging is especially
useful for unstable models that can react drastically to small changes in the
training data.

% ┌────────────────────────────────────────────────────────────────────────────┐
% │ Regression and Classification                                              │
% └────────────────────────────────────────────────────────────────────────────┘

\subsubsection{Regression and Classification}\label{subsubsec:regclas}
\addcontentsline{toc}{subsection}{Regression and Classification}
While the variety of mathematical models seems almost endless, the models
employed in the publications that comprise this dissertation belong
predominantly to the field of machine learning. These algorithms encompass
models that use sample or training data to learn and make predictions \citep{%
Mitchell}. Machine learning can be divided into three main approaches,
unsupervised and supervised learning, and reinforcement learning. This third
category covers the behaviour of intelligent agents that interact with the
environment and is of only marginal interest to the topics at hand \citep{%
Joshi2021a}.

Unsupervised \marginnote{Supervised and Unsupervised Learning}\label{%
mar:supunsup} learning differs from supervised learning by the type of
training data required \citep{Hinton1999}. Unsupervised  models are not
reliant on labelled data, meaning data that has been annotated by humans or
other models. Instead, unsupervised learning aims to build an internal
representation of the space it operates in and capture previously known
patterns according to that representation. Some prominent examples of
unsupervised learning include clustering \citep{Rokach2005}, dimensionality
reduction \citep{VanDerMaaten2009}, and outlier detection \citep{Hawkins1980}.
Supervised learning, on the other hand, does require labelled data. Its goal
is to find a function that maps input variables to output variables as best as
possible, or in other words, to train a model so that it predicts an outcome
as best as possible, given a set of corresponding observations \citep{%
Mohri2018}. Some prominent examples for supervised learning include Bayesian 
inference \citep{Gelman2014}, decision trees \citep{Kaminski2018}, and support 
vector machines \citep{Cortes1995}. Most of the methods discussed in detail
prior to the included publications are supervised.

Supervised learning is commonly further subdivided into two fields of
application: regression and classification. While regression predicts a
numerical (i.e., continuous) outcome, classification predicts discrete class
labels \citep{James2009}.

% ┌────────────────────────────────────────────────────────────────────────────┐
% │ Peculiarities of Biological Data                                           │
% └────────────────────────────────────────────────────────────────────────────┘

\subsubsection{Peculiarities of Biological Data}\label{subsubsec:biodata}
\addcontentsline{toc}{subsection}{Peculiarities of Biological Data}
As an empirical and descriptive discipline, the life sciences, and biology, in
particular, are founded on data. The nature of this information is thus vital
to all scientific enquiries in the field. Following the introductions by
\citet{Jagadish2003} and \citet{Wooley2006}, biological data can be broadly
classified into the following types.
\medbreak

\noindent
Biological\marginnote{Spatial Data}\label{mar:dataspatial} systems, from
strands of DNA in the nucleus of a cell to animal migrations taking place over
thousands of kilometres, are three-dimensional in nature and therefore carry
spatial information. Measuring and encoding the differences between one region
and another in machine-readable form is thus instrumental. Scalar and vector
fields can be seen as an extension of this, as they encode phenomena that are
continuous in space, such as biochemical properties like concentration
gradients or population densities.
\medbreak

\noindent
Sequence data\marginnote{Sequence Data}\label{mar:dataseq} is currently one of
the most abundant forms of biological information and arguable responsible for
the vast majority of progress in the field over the last decades. The amount
of available DNA and RNA sequences is also increasing ever more quickly with
the development of novel technology, such as single-cell sequencing \citep{%
Wang2015} and spatial genomics \citep{Turczyk2020}. While these two techniques
further explore the genetic organisation on the level of individual cells,
another approach called metagenomics analyses all sequences present in an
ecosystem \citep{Venter2004,Hugenholtz2008}. Sequence data can be generalised
as strings representing the DNA or RNA bases, including gaps.
\medbreak

\noindent
Within\marginnote{Patterns}\label{mar:datapatterns} DNA and RNA sequences lie
patterns that represent functional units, such as genes in the genome,
functional elements like promoters and enhancers \citep{Kim2015a}, or
restriction sites \citep{Smith1976}. Patterns can be encoded as context-free
grammars \citep{Hopcroft2001}, Hidden Markov Models (HMMs) \citep{Stamp2021},
or regular expressions \citep{Wang2019}.
\medbreak

\noindent
Another\marginnote{Models}\label{mar:datamodels} type of information that can
be regarded as biological data are the mathematical models created to analyse
experimental measurements. With the increasing number of publications, the
models contained therein, their structure and parameters, need to be machine-
readable to facilitate comparisons.
\medbreak

\noindent
Images\marginnote{Images}\label{mar:dataimages} originating from electron and
optical microscopy, radiography, and other methods, as well as videos, are
another type of data that is especially difficult to convert into a machine-%
readable form. While storing the raw data digitally is trivial, extracting the
features contained in the recordings is not and has spawned the
interdisciplinary field of computer vision \citep{Ballard1982}.
\medbreak

\noindent
Relationships\marginnote{Graphs}\label{mar:datagraphs} such as biochemical and
signalling pathways and phylogenetic trees can be represented as graphs, along
with gene regulatory networks and laboratory workflows. Even sequence data can
be presented in a graph structure to efficiently encode DNA and RNA sequences
variability between individuals \citep{Novak2017}. Geometric arrangements such
as the three-dimensional shape of proteins that governs their docking
behaviour can also be rendered in graph-form.
\medbreak

\noindent
Finally,\marginnote{Prose}\label{mar:dataprose} the literature itself is a
form of data, and the annotations, hypotheses, and inferences stated in
continuous text are difficult to translate into machine-readable form, as well
\citep{Balyan2017}.
\bigbreak

\noindent
As should become clear from this diverse list, biological data can be very
heterogeneous, which can further complicate its analysis, as models may need
to be found which can integrate the different data types. Biological data also
originates, in most cases, from laboratory experiments. This has the
consequence that equipment- and protocol-dependent biases are almost
guaranteed to be present. Even the person performing the experiment can be a
confounding factor. It is thus highly unusual that experimental results from
different laboratories agree. A promising remedy for this is zero-sum
regression by \citet{Altenbuchinger2017}, an extension to conventional linear
regression that has also been adapted to logistic regression \citep{%
Kleinbaum2002} and Cox proportional hazard regression \citep{Fox2002}.
Zero-sum regression, in its simplest from, enforces the zero-sum constraint
on a log-linear model, meaning a linear model on log-transformed data where
the choice of the reference point can result in sample-wise shifts.
\begin{align*}
  \intertext{Assume\marginnote{Zero-Sum Regression}\label{mar:zerosum} again a
  model with many parameters $\theta_{j}$ that should predict an outcome
  $y_{i}$ from predictor $x_{i}$ with the form}
  \widehat{y}_{i}=&\theta_{0}+\sum_{j=1}^{n}\theta_{j}
    \operatorname{log}\!\left(y_{i}\,x_{ij}\right)\\
  \widehat{y}_{i}=&\theta_{0}+
    \sum_{j=1}^{n}\theta_{j}\operatorname{log}\!\left(y_{i}\right)+
    \sum_{j=1}^{n}\theta_{j}\operatorname{log}\!\left(x_{ij}\right)\\
  \widehat{y}_{i}=&\theta_{0}+
    \operatorname{log}\!\left(y_{i}\right)
    {\color{s-cyan}\sum_{j=1}^{n}\theta_{j}}+
    \sum_{j=1}^{n}\theta_{j}\operatorname{log}\!\left(x_{ij}\right)
\end{align*}
By restricting the sum of coefficients (marked in {\color{s-cyan}blue}) to
zero, the linear model is made insensitive to the choice of the reference
point. Example reference points include the amount of DNA or RNA included in
an experiment or the number of cells. Zero-sum regression assumes that any
chosen reference point can be suboptimal. Thus, a model and the resulting
biological interpretation should not rely on it, if possible. The systematic
differences between experimental conditions are modelled separately and can
thus be removed. The intuition behind the method is that, in high-dimensional
space, a subspace is found that lies orthogonal to the unwanted shift in the
data. Thereby, the subspace becomes invariant to the systematic differences.
% Zero-sum regression also does not compromise predictive performance

Other pitfalls of biological data include its volume, variance, and range. Due
to being measurements of inherently noisy phenomena, biological data is usually
generated in replicates. This makes it possible to attribute parts of the
observed variance to experimental conditions, such as batch effects, while the
remainder derives from the phenomenon itself. However, this also means that the
data volume is multiplied by the number of replicates. Because sets of raw
data generated by modern methods can easily reach several hundreds of gigabytes
in size, these measurements can become challenging to handle without
appropriate computational resources. Biological phenomena also tend to span
several orders of magnitude, for example, in the case of transcript counts
associated with individual genomic loci. This complicates matters primarily due
to the human factor involved since humans are generally ill-equipped to think
analytically in logarithmic terms, even though our senses perceive stimuli on a
logarithmic scale \citep{Sun2012a}.

Finally,\marginnote{Curse of Dimensionality}\label{mar:dimensionality}
biological data also tends to suffer from the curse of dimensionality, a term
coined by  Richard E. Bellman \citep{Bellman1957,Bellman1961}. In machine
learning, dimensions are synonymous with features, and the curse of
dimensionality refers to the pros and cons of a data set with many features; a
high-dimensional data set. On the one hand, having many features can be a
blessing when it comes to separating data into distinct classes, as points that
are difficult, if not impossible, to separate clearly in low dimensions can
become easy to separate in higher dimensions. However, on the other hand, when
the dimensionality of Euclidean space increases, the distance between points in
the space increases, too, as it is proportional to the square root of the
number of dimensions \citep{Tabak2014}. This has the consequence that, with
increasing dimensions, Euclidean space becomes vast, and the data becomes
sparse. Dimensionality reductions methods like PCA \citep{Pearson1901}, t-SNE
\citep{Hinton}, UMAP \citep{McInnes2018}, or Autoencoders \citep{Kramer1991}
can serve as a remedy for this problem.

% ┌────────────────────────────────────────────────────────────────────────────┐
% │ Method Overview                                                            │
% └────────────────────────────────────────────────────────────────────────────┘

\section{Method Overview}\label{sec:methoverview}
As many statistical methods are used in more than one instance in the included
publications, the following section describes the main methods of interest.

% Brown–Forsythe test
% Fisher's exact test
% Mann-Whitney U test

% correlation analysis

% decision trees: conditional inference trees
% random forest: mean decrease in accuracy

% generalised linear model

% de bruijn genome graph

% differential expression analysis

% maximum likelihood estimation







