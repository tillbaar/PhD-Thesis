% ┌────────────────────────────────────────────────────────────────────────────┐
% │ DOCUMENT CLASS AND PAGE GEOMETRY                                           │
% └────────────────────────────────────────────────────────────────────────────┘

\documentclass[
 paper=a4                       % use standard A4 paper size
,fontsize=10pt                  % common are 10pt, 11pt, or 12pt
,BCOR=1.0cm                     % defines the binding correction
,twoside=true                   % defines a left-page / right-page document
,DIV=8                          % defines the page layout
,parskip=false                  % removes vertical distance after paragraphs
,numbers=noendperiod            % no dot after last chapter number
,toc=bib                        % bibliography in ToC
,toc=listof                     % list of figures and list of tables in ToC
,version=last                   % use latest KOMA-Script version
,captions=rightbeside           % beside captions on inner side of the figure
,captions=tableheading          % table captions on top of the table
,usegeometry                    % makes KOMA nice play with the geometry package
,chapterprefix                  % this makes the chapter heading magic work
]{scrbook}
  \pagestyle{myheadings}        % allows for personalized header and footer
  \setcapindent{0cm}            % removes the indentation of captions
  \setkomafont{caption}{\small} % defines font size for captions
  \setparsizes{4.0em}{0pt}{1.0em plus 1.0fil} % sets paragraph indetation to 4em

% A common measure for how many letter should fit in a single line to maximize
% readability is 2.5 alphabets, or 66 letters (for the latin alphabet). In "The
% Elements of Typographic Style", Robert Bringhurst gives a table of preferred
% line lengths for different fonts. A least squares optimization fit of the
% table data reveales that it can be approximated quite well with the following:
% see tex.stackexchange.com/questions/59626
\newcommand*\GetTextWidth[3][\normalfont]{{#1
  \settowidth{#2}{abcdefghijklmnopqrstuvwxyz}
  \setlength{#2}{0.03193#2}
  \addtolength{#2}{0.44961pt}
  \setlength{#2}{#3#2}
  \global#2=#2}}

\newlength\bringhurstwdt
\GetTextWidth{\bringhurstwdt}{66}

\usepackage{nextpage}                             % provides additional commands

\usepackage{geometry}                             % provides page layout control
\geometry{
  twoside,                                        % defines a left / right page
  footskip=\dimexpr(\footskip)*3,                 % threefold footer distance
  textwidth=\bringhurstwdt,                       % textwidth calculated above
  showframe=false                                 % for debugging: set true
}

\usepackage[automark, headsepline]{scrlayer-scrpage}
  \ohead{\headmark}                               % outer header contains info
  \automark[section]{section}                     % header into is section
  \setkomafont{pagehead}{\upshape\sffamily}       % header font is sans serif

% The following magic creates the cool vertical bar page numbers
% see www.komascript.de/howto_paginationseparationinfoot
\newcommand*{\pagenumberseparatorline}{%
  \raisebox{0pt}[0pt][0pt]{%
    \rule[-\dimexpr\paperheight-(1in+\topmargin+\headheight+\headsep+
    \textheight+\footskip)]
    {0.5pt}
    {\dimexpr\paperheight-(1in+\topmargin+\headheight+\headsep+
    \textheight+\footskip)+\footheight-\dp\strutbox}%
  }%
}
\lefoot*{%
  \makebox[0pt][r]{%
    \pagemark
    \makebox[\dp\strutbox][l]{\nobreakspace\pagenumberseparatorline}%
  }%
}
\rofoot*{%
  \makebox[0pt][l]{%
    \makebox[\dp\strutbox][r]{\pagenumberseparatorline\nobreakspace}%
    \pagemark
  }%
}

\usepackage{marginnote}                           % provides better margin notes
  \renewcommand*{\marginfont}{\color{s-cyan}\sffamily}

% ┌────────────────────────────────────────────────────────────────────────────┐
% │ COLOURS AND GRAPHICS                                                       │
% └────────────────────────────────────────────────────────────────────────────┘

\usepackage{chngcntr}
  \counterwithout{figure}{chapter}                % figures are cont. numbered
  \counterwithout{table}{chapter}                 %  tables are cont. numbered

\usepackage{graphicx}                             % provides figure management
% \graphicspath{{./Figures/}}                     % defines figure location

\usepackage[dvipsnames, table]{xcolor}            % provides color management
% \definecolor{c_white}{HTML}{f2ece4}             % light yellow for page color
% \definecolor{c_gold}{HTML}{dbb677}              % gold for rules and bullets
% \definecolor{c_red}{HTML}{58180d}               % red for section titles
% \definecolor{c_green}{HTML}{e0e5c1}             % green for table rows
\usepackage[prefix=s-]{xcolor-solarized}          % provides solarized color pal

\usepackage[skins]{tcolorbox}                     % provides colored boxes
  \newtcbox\tcbib{frame empty, hbox, on line, sharp corners,
  colback=s-base01!10, right=1ex, left=1ex, top=-1pt,
  bottom=\dimexpr-\dp\strutbox+1pt}

\usepackage{calc}                                 % provides adv. calculcations
\usepackage{tikz}                                 % provides vector drawings
  \tikzset{every picture/.style={/utils/exec={\sffamily}}}
  \usetikzlibrary{calc}                           % for coordinate calculation.

\usepackage{epstopdf}
  \epstopdfsetup{update}

% ┌────────────────────────────────────────────────────────────────────────────┐
% │ FONTS AND TYPES                                                            │
% └────────────────────────────────────────────────────────────────────────────┘

\usepackage[main=english, german]{babel}          % provides language encoding
\usepackage[T1]{fontenc}                          % provides font encoding
\usepackage[utf8]{inputenc}                       % provides input encoding
\usepackage{lmodern}                              % provides Latin Modern fonts
\usepackage{microtype}                            % typographical perfection
\usepackage[defaultlines = 2, all]{nowidow}       % prevents widows and orphans
\usepackage{siunitx}                              % typesets physical quantities

% ┌────────────────────────────────────────────────────────────────────────────┐
% │ MATHS                                                                      │
% └────────────────────────────────────────────────────────────────────────────┘

\usepackage{amsmath}
  \renewcommand{\theequation}{\arabic{equation}}
\usepackage{amssymb}
\usepackage{amstext}

\usepackage{aligned-overset}
\usepackage{nicefrac}
\usepackage{cancel}

% ┌────────────────────────────────────────────────────────────────────────────┐
% │ REFERENCES AND BIBLIOGRAPHY                                                │
% └────────────────────────────────────────────────────────────────────────────┘

\usepackage[comma, sort, square]{natbib}          % provides citation management
  \setcitestyle{aysep={}}                         % no sep between Author & Name
\usepackage[
  bookmarks,                                      % create PDF bookmarks
  bookmarksopen=true,                             % unfold bookmarks in viewer
  bookmarksopenlevel=1,                           % level of unfolding (section)
  bookmarksnumbered=true,                         % number PDF bookmarks
  hidelinks,                                      % do not highlight hyperlinks
  colorlinks=true,                                % color hyperlinks
  citecolor=s-base01,                             % color of citations
  linkcolor=black,                                % color of hyperlinks
  urlcolor=s-base01,                              % color of urls
  pdfpagelabels=true,                             % see manual…
  plainpages=false,                               % see manual…
  hyperindex=true,                                % index entries link to text
  linktoc=all                                     % section & page are hyperlink
]{hyperref}                                       % provides automatic hyperlink

\usepackage[nameinlink, noabbrev]{cleveref}       % better ref and link control

% The following magic creates the cool key listings in the bibliography
% see tex.stackexchange.com/questions/152712
\makeatletter
\def\@lbibitem[#1]#2{%
\hypersetup{citecolor=black}
    \if\relax\@extra@b@citeb\relax\else
    \@ifundefined{br@#2\@extra@b@citeb}{}{%
    \@namedef{br@#2}{\@nameuse{br@#2\@extra@b@citeb}}}\fi
  \@ifundefined{b@#2\@extra@b@citeb}{\def\NAT@num{}}{\NAT@parse{#2}}%
  \item[\hfil\hyper@natanchorstart{#2\@extra@b@citeb}
    \sffamily\tcbib{\strut\citealp{#2}}%
  \hyper@natanchorend]%
  \NAT@ifcmd#1(@)(@)\@nil{#2}}
\makeatother

% ┌────────────────────────────────────────────────────────────────────────────┐
% │ TABLES AND LISTINGS                                                        │
% └────────────────────────────────────────────────────────────────────────────┘

\usepackage{tabularx}                             % provides auto table width
\usepackage{multirow}                             % provides column and row span
\usepackage{booktabs}                             % provides better table design

\usepackage{tcolorbox}                            % provides better textboxes

\usepackage{blindtext}                            % provides better lorem ipsum