% \section*{Zusammenfassung}
% \begin{tcolorbox}[arc=0pt, boxrule=0.5pt, colback=white, colframe=s-blue,
%   rightrule=0pt, toprule=0pt, top=0pt, bottom=\ht\strutbox]
% \blindtext[2]
% \end{tcolorbox}

% \pagebreak

\null
\vspace{\dimexpr2.3\baselineskip+1\parskip\relax}

\begingroup
  \fontsize{22}{26.4}\color{s-blue}\sffamily\bfseries
  \noindent\raggedleft Abstract\par
  \vspace*{-.5\baselineskip}
  \noindent\rule{\textwidth}{.5pt}\par\nobreak
\endgroup
\vspace{\baselineskip}

\noindent In this cumulative dissertation, statistical models for regression
are discussed in light of high-dimensional, biological data. The dissertation
includes three publications:
\medbreak

\noindent\textcolor{s-base01}{RNA transcription and degradation of Alu
retrotransposons depends on sequence features and evolutionary history}
examines Alu elements, RNA retrotransposons in the human genome. Their RNA
metabolism is poorly understood, and the source of Alu transcripts is still
unresolved. We have conducted a transcription shutoff experiment and metabolic
RNA labelling to shed further light on the life cycle of Alu transcripts. We
furthermore present a novel statistical test for detecting expression
quantitative trait loci relying on k-mer sequence representation.
\medbreak

\noindent\textcolor{s-base01}{Endoscopic hemostasis makes the difference:
Angiographic treatment in patients with lower gastrointestinal bleeding} uses
retrospective study data from patients receiving either endoscopic or
angiographic treatment for lower gastrointestinal bleeding. While a majority
of patients can be treated successfully with the usually preferred endoscopic
method, in some cases, angiography is required to achieve hemostasis. Using
conditional inference trees, we construct a decision tree model predicting if
a patient should receive angiographic treatment.
\medbreak

\noindent\textcolor{s-base01}{Genetic instability and recurrent \textit{MYC}
amplification in \textit{ALK}-translocated NSCLC: a central role of
\textit{TP53} mutations} investigates a molecular subtype of lung cancer
exhibiting rearrangements of the \textit{ALK} gene. This cancer type often
resists treatments, and no reliable biomarker to identify patients at risk for
relapse is known. Analysing biopsy and cell culture data, we find that
mutations in the \textit{TP53} gene can lead to chromosomal instability and
thus the amplification of known cancer genes. This, in turn, grants cancer
cells a proliferative advantage compared to the wild-type, providing a new
approach for diagnosis and treatment.